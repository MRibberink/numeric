\documentclass[12pt]{article}

../../lab3/pdfs/def.tex

\usepackage{fullpage}

\pagestyle{empty}

\begin{document}

\begin{flushright}
\today
\end{flushright}

\section*{Assignment Set for Laboratory 1}
{\it ATSC 409: Hand-in answers to questions 1, 2 and 3.\\
EOSC 511/ATSC 506: Hand-in answers to questions 1, 3 and 4.}


\begin{enumerate}
\item Given the following four (x,y) points  (-5,-1), (0,0), (5,1), (8,4) find the y-value at x=3 using
\begin{enumerate}
\item Linear Interpolation
\item Cubic Interpolation
\end{enumerate}
\item Given the equation
\begin{equation}
\dydt = y(y+t)
\end{equation}
write down
\begin{enumerate}
\item forward Euler difference formula
\item backward Euler difference formula
\item centered difference formula
\end{enumerate}
\item The equation
\begin{equation}
\dydt + c \dydx = 0,\ y = \cos(x) \Mathat t=0,\ \dydt = c \sin(x) \Mathat t=0
\end{equation}
has a solution $y=\cos(x-ct)$.
\begin{enumerate}
\item Expand both derivatives as centred differences.
\item Show that the algebraic solution is an exact solution of the difference formula if we choose $\Delta x = c \Delta t$.
\end{enumerate}
\item Given
\begin{equation}
\dydt = -\alpha y,\ y = 1 \Mathat t=0
\end{equation}
\begin{enumerate}
\item Show that the forward Euler method gets a smaller answer than the backward Euler method for all $t > 0$, provided that $0 < \alpha^2 \Delta t^2 < 1$.
\item Solve the equation analytically.
\item Show that the forward Euler always under-estimates the answer provided that $\alpha \Delta t < 1 \Mathand \alpha \Delta t \ne 0$.
\end{enumerate}\end{enumerate}

\end{document}

%%% Local Variables:
%%% mode: latex
%%% TeX-master: t
%%% End:
